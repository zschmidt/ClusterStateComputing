\documentclass[onecolumn]{IEEEtran11}


\usepackage[usenames,dvipnames]{xcolor} % for colored text
\usepackage[normalem]{ulem} % for strickthrough text

% quantum macros

\def\01{\{0,1\}}
\def\eps{\epsilon}
\newcommand{\half}{{\frac{1}{2}}}
\newcommand{\set}[1]{{\left\{#1\right\}}}
\newcommand{\ksubsets}{{n \choose k}}
\newcommand{\jsubsets}{{n \choose j}}
\newcommand{\Prob}{{\mathbf{Pr}}}
\newcommand{\tinyspace}{\mspace{1mu}}
\newcommand{\microspace}{\mspace{0.5mu}}
\newcommand{\op}[1]{\operatorname{#1}}

\newcommand{\norm}[1]{\left\lVert\tinyspace#1\tinyspace\right\rVert}
\newcommand{\snorm}[1]{\lVert\tinyspace#1\tinyspace\rVert}
\newcommand{\abs}[1]{\left\lvert\tinyspace #1 \tinyspace\right\rvert}
\newcommand{\ceil}[1]{\left\lceil #1 \right\rceil}
\newcommand{\floor}[1]{\left\lfloor #1 \right\rfloor}
\def\iso{\cong}
\newcommand{\defeq}{\stackrel{\smash{\text{\tiny def}}}{=}}
\newcommand{\tr}{\operatorname{tr}}
\newcommand{\rank}{\operatorname{rank}}
\renewcommand{\det}{\operatorname{Det}}
\newcommand{\im}{\operatorname{Im}}
\renewcommand{\t}{{\scriptscriptstyle\mathsf{T}}}
\newcommand{\ip}[2]{\left\langle #1 , #2\right\rangle}
\newcommand{\ipp}[1]{\left\langle #1 \right\rangle}
\newcommand{\sip}[2]{\langle #1 | #2\rangle}

\def\({\left(}
\def\){\right)}
\def\I{\mathsf{id}}

\newcommand{\fid}{\operatorname{F}}
\newcommand{\setft}[1]{\mathrm{#1}}
\newcommand{\lin}[1]{\setft{L}\left(#1\right)}
\newcommand{\density}[1]{\setft{Dens}\left(#1\right)}
\newcommand{\unitary}[1]{\setft{U}\left(#1\right)}
\newcommand{\trans}[1]{\setft{T}\left(#1\right)}
\newcommand{\herm}[1]{\setft{Herm}\left(#1\right)}
\newcommand{\pos}[1]{\setft{Pos}\left(#1\right)}
\newcommand{\pd}[1]{\setft{Pd}\left(#1\right)}
\newcommand{\sphere}[1]{\mathcal{S}\!\left(#1\right)}
\newcommand{\opset}[3]{\setft{#1}_{#2}\!\left(#3\right)}
\newcommand{\ot}{\otimes}

\def\complex{\mathbb{C}}
\def\real{\mathbb{R}}
\def\natural{\mathbb{N}}
\def\integer{\mathbb{Z}}

\def\<{\langle}
\def\>{\rangle}
\def \lket {\left|}
\def \rket {\right\rangle}
\def \lbra {\left\langle}
\def \rbra {\right|}
\newcommand{\ket}[1]{\lket\microspace #1 \microspace\rket}
\newcommand{\bra}[1]{\lbra\microspace #1 \microspace\rbra}
\newcommand{\ketbra}[1]{\lket\microspace #1 \rangle \langle #1 \microspace\rbra}


\def\X{\mathcal{X}}
\def\Y{\mathcal{Y}}
\def\Z{\mathcal{Z}}
\def\W{\mathcal{W}}
\def\A{\mathcal{A}}
\def\B{\mathcal{B}}
\def\V{\mathcal{V}}
\def\U{\mathcal{U}}
\def\C{\mathcal{C}}
\def\D{\mathcal{D}}
\def\H{\mathcal{H}}
\def\E{\mathcal{E}}
\def\F{\mathcal{F}}
\def\M{\mathcal{M}}
\def\R{\mathcal{R}}
\def\P{\mathcal{P}}
\def\Q{\mathcal{Q}}
\def\S{\mathcal{S}}
\def\T{\mathcal{T}}
\def\K{\mathcal{K}}
\def\yes{\text{yes}}
\def\no{\text{no}}
\def\onevec{\vec{\mathbf{1}}}

\newcommand{\trnorm}[1]{\norm{#1}_{\tr}}
\newcommand{\trnormb}[1]{{\big\| #1 \big\|}_{\rm tr}}
\newcommand{\trdist}[1]{ \left | #1 \right |_{\rm tr}}
\newcommand{\uniform}[1]{\mathcal{U}_{#1}}

\def\defeq{\stackrel{\small \textrm{def}}{=}}


\usepackage{qcircuit}
\usepackage{amsmath}
\usepackage{epsfig}
\usepackage[T1]{fontenc}
\usepackage{graphicx}
\def\BibTeX{{\rm B\kern-.05em{\sc i\kern-.025em b}\kern-.08em
    T\kern-.1667em\lower.7ex\hbox{E}\kern-.125emX}}

\oddsidemargin -15pt
\evensidemargin -15pt
\leftmargin 0 pt
\topmargin -30pt
\textwidth = 6.9 in
\textheight = 9.0 in

\newcommand{\itembase}{\setlength{\itemsep}{0pt}}
\newcommand{\eg}{{\it e.g., }}
\newcommand{\ie}{{\it i.e., }}

% All the pretty colors go here
\definecolor{myGreen}{rgb}{0,1,0}
\definecolor{myRed}{rgb}{1,0,0}
\newcommand{\clb}{\color{blue}}
\newcommand{\clr}{\color{myRed}}
\newcommand{\clg}{\color{myGreen}}
\newcommand{\clbl}{\color{black}}

\begin{document}


\title{{\Large \bf Cluster State Quantum Computing}\\ {\normalsize CIS:410/510 Midterm Report, Spring 2016}}
\author{
Dileep Reddy, Mayra Amezcua, Zach Schmidt \\
{\em dileep@uoregon.edu, mamezcua@cas.uoregon.edu, zschmidt@cs.uoregon.edu }
}
\maketitle

\section{4-node Cluster State}
\subsection{Linear Cluster State}
We start with four qubits in the $\ket{+}$ state and apply a CZ gate on the first two qubits to entangle them. 
%\[\begin{array}{cccc}
%\bullet&\bullet&\bullet&\bullet\\
%\ket{+}_1&\ket{+}_2&\ket{+}_3&\ket{+}_4\end{array}\]

\[\Qcircuit @C=1em @R=1.6em{
\lstick{\ket{+}}&\ctrl{1}&\qw\\
\lstick{\ket{+}}&\gate{CZ}&\qw\\
\lstick{\ket{+}}&\qw&\qw\\
\lstick{\ket{+}}&\qw&\qw}\]\vspace{3ex}

\[CZ_{12}\ket{+}_1\ket{+}_2\ket{+}_3\ket{+}_4=\left(\frac{\ket{0}_1\ket{+}_2+\ket{1}_1\ket{-}_2}{\sqrt2}\right)\ket{+}_{3}\ket{+}_{4}=\left(\frac{\ket{+}_1\ket{0}_2+\ket{-}_1\ket{1}_2}{\sqrt2}\right)\ket{+}_{3}\ket{+}_{4}\]\vspace{1ex}

Now we apply a CZ gate to qubits 2 and 3. 

\[\Qcircuit @C=1em @R=1.6em{
\lstick{\ket{+}}&\ctrl{1}&\qw&\qw&\qw\\
\lstick{\ket{+}}&\gate{CZ}&\qw&\ctrl{1}&\qw\\
\lstick{\ket{+}}&\qw&\qw&\gate{CZ}&\qw\\
\lstick{\ket{+}}&\qw&\qw&\qw&\qw}\]\vspace{3ex}

\[CZ_{23}\left(\frac{\ket{+}_1\ket{0}_2\ket{+}_{3}+\ket{-}_1\ket{1}_2\ket{+}_{3}}{\sqrt2}\right)\ket{+}_{4}=\left(\frac{\ket{+}_1\ket{0}_2\ket{+}_{3}+\ket{-}_1\ket{1}_2\ket{-}_{3}}{\sqrt2}\right)\ket{+}_{4}\]
\[=\frac{1}{\sqrt2}\left[(\ket{+}_1\ket{0}_2+\ket{-}\ket{1})\ket{0}_{3}+(\ket{+}_1\ket{0}_2-\ket{-}\ket{1})\ket{1}_{3}\right]\ket{+}_{4}\]\vspace{1ex}

Finally we apply one last CZ gate on the last two qubits.

\[\Qcircuit @C=1em @R=1.6em{
\lstick{\ket{+}}&\ctrl{1}&\qw&\qw&\qw&\qw&\qw\\
\lstick{\ket{+}}&\gate{CZ}&\qw&\ctrl{1}&\qw&\qw&\qw\\
\lstick{\ket{+}}&\qw&\qw&\gate{CZ}&\qw&\ctrl{1}&\qw\\
\lstick{\ket{+}}&\qw&\qw&\qw&\qw&\gate{CZ}&\qw}\]\vspace{3ex}

\[CZ_{34}\frac{1}{\sqrt2}\left[(\ket{+}_1\ket{0}_2+\ket{-}\ket{1})\ket{0}_{3}\ket{+}_{4}+(\ket{+}_1\ket{0}_2-\ket{-}\ket{1})\ket{1}_{3}\ket{+}_{4}\right]\]
\[=\frac{1}{\sqrt2}\left[(\ket{+}_1\ket{0}_2+\ket{-}\ket{1})\ket{0}_{3}\ket{+}_{4}+(\ket{+}_1\ket{0}_2-\ket{-}\ket{1})\ket{1}_{3}\ket{-}_{4}\right]\]
\[=\frac{1}{\sqrt2}(\ket{+}_{1}\ket{0}_2\ket{+}_3+\ket{-}_{1}\ket{1}_2\ket{-}_3)\ket{0}_4+\frac{1}{\sqrt2}(\ket{+}_{1}\ket{0}_2\ket{-}_3+\ket{-}_{1}\ket{1}_2\ket{+}_3)\ket{1}_4\]

\subsection{T-shaped Cluster State}


\end{document}