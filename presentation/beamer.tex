
\documentclass[8pt,mathserif]{beamer}
\usetheme{Warsaw}
\usecolortheme{whale}
\usepackage{times}  % fonts are up to you
\usepackage{graphicx}
\usepackage{verbatim}
\usepackage{amsmath}
\usepackage{color}
\usepackage{listings}
\usepackage{lmodern}
\usepackage{multicol}
\usepackage[T1]{fontenc}
\usepackage[style=numeric-comp]{biblatex}
%\usepackage{bibentry}
\setbeamercolor{background canvas}{bg=gray}

\setbeamertemplate{footline}[page number]{}

%\addbibresource{TMI_bib.bib}
%\bibliography{TMIbib.bib}

\newenvironment<>{varblock}[2][.9\textwidth]{%
  \setlength{\textwidth}{#1}
  \begin{actionenv}#3%
    \def\insertblocktitle{#2}%
    \par%
    \usebeamertemplate{block begin}}
  {\par%
    \usebeamertemplate{block end}%
  \end{actionenv}}

\makeatletter
\pgfdeclareverticalshading[black,bg]{bmb@shadow}{200cm}{%
  color(0bp)=(black!25); color(4bp)=(black!50!bg); color(8bp)=(black!50!bg)}
\pgfdeclareradialshading[black,bg]{bmb@shadowball}{\pgfpointorigin}{%
  color(0bp)=(black!50!bg); color(4bp)=(black!25)}
\pgfdeclareradialshading[black,bg]{bmb@shadowballlarge}{\pgfpointorigin}{%
  color(0bp)=(black!50!bg); color(4bp)=(black!50!bg); color(8bp)=(black!25)}

\def\insertsectionnavigation#1{%
  \hbox to #1{\vbox{{\usebeamerfont{section in head/foot}%
     \usebeamercolor[fg]{section in head/foot}%
     \def\slideentry##1##2##3##4##5##6{}%
     \def\sectionentry##1##2##3##4##5{%
       \ifnum##5=\c@part%
       \def\insertsectionhead{##2\hskip1em}%
       \def\insertsectionheadnumber{##1}%
       \def\insertpartheadnumber{##5}%
         \hyperlink{Navigation##3}{%
             \ifnum\c@section=##1%
               {\usebeamertemplate{section in head/foot}}%
             \else%
               {\usebeamertemplate{section in head/foot shaded}}%
             \fi%
         }\par
       \fi}%
       \parbox[c][0cm][c]{.5\paperwidth}{%
       \begin{multicols}{2}
       \dohead
       \end{multicols}}\space}
     }%
  \hfil}%
}

\def\insertsubsectionnavigation#1{%
  \hbox to #1{%
    \vbox{{%
      \usebeamerfont{subsection in head/foot}\usebeamercolor[fg]{subsection in head/foot}%
      \vskip0.5625ex%
      \beamer@currentsubsection=0%
      \def\sectionentry##1##2##3##4##5{}%
      \def\slideentry##1##2##3##4##5##6{\ifnum##6=\c@part\ifnum##1=\c@section%
        \ifnum##2>\beamer@currentsubsection%
        \beamer@currentsubsection=##2%
        \def\insertsubsectionhead{##5}%
        \def\insertsectionheadnumber{##1}%
        \def\insertsubsectionheadnumber{##2}%
        \def\insertpartheadnumber{##6}%
        \beamer@link(##4){%
              \ifnum\c@subsection=##2%
                {\usebeamertemplate{subsection in head/foot}}%
              \else%
                {\usebeamertemplate{subsection in head/foot shaded}}%
              \fi\hfill}\par
        \fi\fi\fi}%
       \hspace*{0.5em}\parbox[c][0cm][c]{\dimexpr.5\paperwidth-1em\relax}{%
       \begin{multicols}{2}
       \dohead\vskip0.5625ex\end{multicols}
       }\space
   }\hfil
}}}

\setbeamertemplate{navigation symbols}{}
%\setbeamertemplate{blocks}[rounded][shadow=false]
%\setbeamertemplate{title page}[default][colsep=-4bp,rounded=true]
\setbeamertemplate{frametitle}[default][colsep=-4bp,rounded=false,shadow=false]

\setbeamertemplate{headline}
{%
  \leavevmode\@tempdimb=2.4375ex%
  \ifnum\beamer@subsectionmax<\beamer@sectionmax%
    \multiply\@tempdimb by\beamer@sectionmax%
  \else%
    \multiply\@tempdimb by\beamer@subsectionmax%
  \fi%
  \ifdim\@tempdimb>0pt%
    \advance\@tempdimb by 1.125ex%
    \begin{beamercolorbox}[wd=.5\paperwidth,ht=0.5\@tempdimb,dp=2ex]{section in head/foot}%
      \vbox to0.5\@tempdimb{\vfill\insertsectionnavigation{.5\paperwidth}\vfill}%
    \end{beamercolorbox}%
    \begin{beamercolorbox}[wd=.5\paperwidth,ht=0.5\@tempdimb,dp=2ex]{subsection in head/foot}%
      \vbox to0.5\@tempdimb{\vfill\insertsubsectionnavigation{.5\paperwidth}\vfill}%
    \end{beamercolorbox}%
  \fi%
}
\makeatother

\def\bra#1{\mathinner{\langle{#1}|}}
\def\ket#1{\mathinner{|{#1}\rangle}}
\newcommand{\braket}[2]{\langle #1|#2\rangle}
\def\Bra#1{\left<#1\right|}
\def\Ket#1{\left|#1\right>}

% for themes, etc.
%\mode<presentation>
%{ \usetheme{boxes} }


%\usepackage{beamerposter}

% these will be used later in the title page
\title{\bf{Cluster-State Quantum Computing}}
\author{Mayra Amezcua, Dileep V. Reddy, Zach Schmidt}
\date{\today}


% note: do NOT include a \maketitle line; also note that this title
% material goes BEFORE the \begin{document}

% have this if you'd like a recurring outline
%\AtBeginSection[]  % "Beamer, do the following at the start of every section"
%{
%\begin{frame}<beamer> 
%\frametitle{Outline} % make a frame titled "Outline"
%\tableofcontents[currentsection]  % show TOC and highlight current section
%\end{frame}
%}
\newcommand\Fontvi{\fontsize{7}{7.2}\selectfont}

\begin{document}



\usebackgroundtemplate{\includegraphics[width=\paperwidth]{figures/fio_bg1.png}}
\begin{frame}
\begin{center}
Course Project, Spring 2016
\maketitle

{\bf CIS410/510 Introduction to Quantum Information Theory}\\
Lecturer: Prof. Xiaodi Wu\\
Computer and Information Science, University of Oregon
\end{center}
\end{frame}

\begin{frame}
{Table of Contents}
\begin{multicols}{2}
\tableofcontents
\end{multicols}
\end{frame}

\begin{frame}{template frame}

  \begin{block}{test block}
    some text
  \end{block}

  \begin{varblock}[4cm]{test varblock}
    Variable block (here 4cm)
  \end{varblock}

  \begin{columns}

    \begin{column}{.5\linewidth}
      \begin{alertblock}{test alert}
        some alert
      \end{alertblock}
    \end{column}

    \begin{column}{.5\linewidth}
      \begin{exampleblock}{test example}
        some example citation \footnotemark
      \end{exampleblock}
    \end{column}\footnotetext[1]{Auth, DV, 123, 2001.}

  \end{columns}

\end{frame}


\section{Motivation}
\begin{frame}
  One-way quantum computing, measurement based quantum computing \\
  As opposed to circuit based quantum computing
\end{frame}

\subsection{Gates through teleportation}
\begin{frame}
  Basic teleporation
\end{frame}

\section{Cluster states}

\subsection{Definition}
\begin{frame}
  
\end{frame}


\subsection{Representations}
\begin{frame}
  
\end{frame}


\subsection{Properties}
\begin{frame}
  
\end{frame}


\section{Universal computation through CS}

\subsection{Linear wire}
\begin{frame}
  
\end{frame}

\subsection{Arbitrary single qubit operations}
\begin{frame}
  Callback to teleportation discussion  
\end{frame}


\subsection{Two qubit operations}
\begin{frame}
  
\end{frame}

\section{Advantages and disadvantages}

\subsection{Parallelizability}
\begin{frame}
  
\end{frame}

\subsection{Experimental implementations}
\begin{frame}
  
\end{frame}

\subsection{Cluster state model as an analysis tool}
\begin{frame}
  
\end{frame}

\end{document}
