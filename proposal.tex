\documentclass[twocolumn]{IEEEtran11}

% quantum macros

\def\01{\{0,1\}}
\def\eps{\epsilon}
\newcommand{\half}{{\frac{1}{2}}}
\newcommand{\set}[1]{{\left\{#1\right\}}}
\newcommand{\ksubsets}{{n \choose k}}
\newcommand{\jsubsets}{{n \choose j}}
\newcommand{\Prob}{{\mathbf{Pr}}}
\newcommand{\tinyspace}{\mspace{1mu}}
\newcommand{\microspace}{\mspace{0.5mu}}
\newcommand{\op}[1]{\operatorname{#1}}

\newcommand{\norm}[1]{\left\lVert\tinyspace#1\tinyspace\right\rVert}
\newcommand{\snorm}[1]{\lVert\tinyspace#1\tinyspace\rVert}
\newcommand{\abs}[1]{\left\lvert\tinyspace #1 \tinyspace\right\rvert}
\newcommand{\ceil}[1]{\left\lceil #1 \right\rceil}
\newcommand{\floor}[1]{\left\lfloor #1 \right\rfloor}
\def\iso{\cong}
\newcommand{\defeq}{\stackrel{\smash{\text{\tiny def}}}{=}}
\newcommand{\tr}{\operatorname{tr}}
\newcommand{\rank}{\operatorname{rank}}
\renewcommand{\det}{\operatorname{Det}}
\newcommand{\im}{\operatorname{Im}}
\renewcommand{\t}{{\scriptscriptstyle\mathsf{T}}}
\newcommand{\ip}[2]{\left\langle #1 , #2\right\rangle}
\newcommand{\ipp}[1]{\left\langle #1 \right\rangle}
\newcommand{\sip}[2]{\langle #1 | #2\rangle}

\def\({\left(}
\def\){\right)}
\def\I{\mathsf{id}}

\newcommand{\fid}{\operatorname{F}}
\newcommand{\setft}[1]{\mathrm{#1}}
\newcommand{\lin}[1]{\setft{L}\left(#1\right)}
\newcommand{\density}[1]{\setft{Dens}\left(#1\right)}
\newcommand{\unitary}[1]{\setft{U}\left(#1\right)}
\newcommand{\trans}[1]{\setft{T}\left(#1\right)}
\newcommand{\herm}[1]{\setft{Herm}\left(#1\right)}
\newcommand{\pos}[1]{\setft{Pos}\left(#1\right)}
\newcommand{\pd}[1]{\setft{Pd}\left(#1\right)}
\newcommand{\sphere}[1]{\mathcal{S}\!\left(#1\right)}
\newcommand{\opset}[3]{\setft{#1}_{#2}\!\left(#3\right)}
\newcommand{\ot}{\otimes}

\def\complex{\mathbb{C}}
\def\real{\mathbb{R}}
\def\natural{\mathbb{N}}
\def\integer{\mathbb{Z}}

\def\<{\langle}
\def\>{\rangle}
\def \lket {\left|}
\def \rket {\right\rangle}
\def \lbra {\left\langle}
\def \rbra {\right|}
\newcommand{\ket}[1]{\lket\microspace #1 \microspace\rket}
\newcommand{\bra}[1]{\lbra\microspace #1 \microspace\rbra}
\newcommand{\ketbra}[1]{\lket\microspace #1 \rangle \langle #1 \microspace\rbra}


\def\X{\mathcal{X}}
\def\Y{\mathcal{Y}}
\def\Z{\mathcal{Z}}
\def\W{\mathcal{W}}
\def\A{\mathcal{A}}
\def\B{\mathcal{B}}
\def\V{\mathcal{V}}
\def\U{\mathcal{U}}
\def\C{\mathcal{C}}
\def\D{\mathcal{D}}
\def\H{\mathcal{H}}
\def\E{\mathcal{E}}
\def\F{\mathcal{F}}
\def\M{\mathcal{M}}
\def\R{\mathcal{R}}
\def\P{\mathcal{P}}
\def\Q{\mathcal{Q}}
\def\S{\mathcal{S}}
\def\T{\mathcal{T}}
\def\K{\mathcal{K}}
\def\yes{\text{yes}}
\def\no{\text{no}}
\def\onevec{\vec{\mathbf{1}}}

\newcommand{\trnorm}[1]{\norm{#1}_{\tr}}
\newcommand{\trnormb}[1]{{\big\| #1 \big\|}_{\rm tr}}
\newcommand{\trdist}[1]{ \left | #1 \right |_{\rm tr}}
\newcommand{\uniform}[1]{\mathcal{U}_{#1}}

\def\defeq{\stackrel{\small \textrm{def}}{=}}



\usepackage{amsmath}
\usepackage{epsfig}
\usepackage[T1]{fontenc}
\usepackage{graphicx}
\def\BibTeX{{\rm B\kern-.05em{\sc i\kern-.025em b}\kern-.08em
    T\kern-.1667em\lower.7ex\hbox{E}\kern-.125emX}}

\oddsidemargin -15pt
\evensidemargin -15pt
\leftmargin 0 pt
\topmargin -30pt
\textwidth = 6.9 in
\textheight = 9.0 in

\newcommand{\itembase}{\setlength{\itemsep}{0pt}}
\newcommand{\eg}{{\it e.g., }}
\newcommand{\ie}{{\it i.e., }}

\begin{document}
\bibliographystyle{IEEE}

\title{\Large \bf Cluster State Computing}
\author{
Dileep Reddy, Maira Amezcua, Zach Schmidt \\
{\em dileep@uoregon.edu, mamezcua@cas.uoregon.edu, zschmidt@cs.uoregon.edu }
}
\maketitle

%\pagestyle{empty}
\begin{abstract}
Computation can be done via measurement only, the outcome of which is dependant on the initial entangled state -- the \textit{cluster state}. This paper will attempt to provide background regarding computation using only measurements, a brief forray into the preparation of cluster states, a discussion of one way quantum computers (1WQC), and the computational power of various configurations of a 1WQC.
\end{abstract}

%\begin{keywords} 
%\end{keywords}

\section{Background}
This paper will attempt to discuss some background on cluster states, a correspondence between 1WQC and the more traditional gate array model, and the computational power of this new model.

\subsection{Cluster States}
A cluster state is characterized by a set of eigenvalue equations, which are are determined by the distribution of the qubits on some lattice\cite{briegel2001persistent}. A method to prepare a one-dimensional cluster state is given in \cite{jorrand2005unifying}, consisting of ``cascading'' Controlled-$Z$ ($C_z$) gates on $n$ qubits, where:

\[
C_z = 
\begin{pmatrix}
  1 & 0 & 0 & 0 \\
  0 & 1 & 0 & 0 \\
  0 & 0 & 1 & 0 \\
  0 & 0 & 0 & -1 \\
 \end{pmatrix}
\]

The action of the $C_z$ gate in the computational basis can be seen to be $\ket{x,y} \to (-1)^{xy}\ket{x,y}$. The set of all single qubit unitaries, coupled with either $C_x$ or $C_z$ has been shown to be universal\cite{brylinski2002universal}. Intuitively, a cluster state can be thought of as a graph where every vertex represents a qubit, and every edge represents the application of a $C_z$ gate to both adjacent vertices. Briegel and Raussendorf show that any quantum logic circuit can be implemented on a cluster state, which demonstrates universality of the proposed scheme\cite{briegel2000measurements}. Nielsen\cite{nielsen108020universal} extended this result to no longer require coherent dynamics, instead relying on a method to teleport quantum gates, and he provided a concise algorithm to do it. 

\subsection{One-Way Quantum Computation}
The unidirectionality of cluster state computation is inherent, due to the fact that the entanglement is progressively consumed at every step. An execution on a one-way quantum computer is a sequence of one-qubit measurements on a cluster state\cite{jorrand2005unifying}.


\subsection{Gate Array Correspondence}
In his analysis of the reducability of 1WQC to the gate array model, Richard Jozsa gives a polynomial time algorithm to perform the conversion between the two computational models\cite{jozsa2006introduction}. 

\subsection{Computational Power and Complexity}
The spacial layout of the graph representation of the cluster state plays a role in the computational power of that state. If a cluster state can be prepared linearly via the cascading $C_z$ technique mentioned above, it can be represented as a  ``one-dimensional'' graph (i.e., some graph $G=(V,E),\ \forall v\in V$, deg$(v)\leq 2$). A linearly prepared cluster state can be efficiently simluated on a classical computer in $O(n\log ^c (1/n))$, where $n$ is the initial number of qubits, and $c$ is the cost of floating point multiplication\cite{nielsen2006cluster}. Though the author consequently dismisses linearly prepared cluster states as a substrate for quantum computation, it would be interesting to know which class of problems they would be able to solve. \\
In general, measurement based models can be polynomial time reduced to the gate array model, and thus have the same power, but they are more easily parallelizable\cite{jozsa2006introduction}.\\
The gate teleportion algorithm\cite{nielsen108020universal} has a time complexity of $O(\log (1/\epsilon)$, where $\epsilon$ is the failure probability. 


%% file citations.bib contains all the biblography
\bibliography{citations}
\end{document}
