\documentclass[twocolumn]{IEEEtran11}

\usepackage{times}
\usepackage{epsfig}
\usepackage[T1]{fontenc}
\usepackage{graphicx}
\def\BibTeX{{\rm B\kern-.05em{\sc i\kern-.025em b}\kern-.08em
    T\kern-.1667em\lower.7ex\hbox{E}\kern-.125emX}}

\oddsidemargin -15pt
\evensidemargin -15pt
\leftmargin 0 pt
\topmargin -30pt
\textwidth = 6.9 in
\textheight = 9.0 in

\newcommand{\itembase}{\setlength{\itemsep}{0pt}}
\newcommand{\eg}{{\it e.g., }}
\newcommand{\ie}{{\it i.e., }}

\begin{document}
\bibliographystyle{IEEE}

\title{\Large \bf Cluster State Computing}
\author{
Dileep Reddy, Maira Amezcua, Zach Schmidt \\
{\em dileep@uoregon.edu, mamezcua@cas.uoregon.edu, zschmidt@cs.uoregon.edu }
}
\maketitle

%\pagestyle{empty}
\begin{abstract}
Computation can be done via measurement only, the outcome of which is dependant on the initial entangled state -- the \textit{cluster state}. This paper will attempt to provide background regarding computation using only measurements, a brief forray into the preparation of cluster states, a discussion of one way quantum computers (1WQC), and the computational power of various configurations of a 1WQC.
\end{abstract}

%\begin{keywords} 
%\end{keywords}

\section{Areas}
This paper will attempt to discuss some background on cluster states, a correspondence between 1WQC and the more traditional gate array model, and the computational power of this new model.

\subsection{Cluster States}
A cluster state is characterized by a set of eigenvalue equations, which are are determined by the distribution of the qubits on some lattice\cite{briegel2001persistent}.
Briegel and Raussendorf show that any quantum logic circuit can be implemented on a cluster state, which demonstrates universality of the proposed scheme\cite{briegel2000measurements}. A method to prepare a one-dimensional cluster state is given in \cite{jorrand2005unifying}, consisting of ``cascading'' $C_z
$ gates on $n$ qubits.

\subsection{Gate Array Correspondence}
In his analysis of the reducability of 1WQC to the gate array model, Richard Jozsa gives a polynomial time algorithm to perform the conversion between the two computational models\cite{jozsa2006introduction}. 

\subsection{Computational Power and Complexity}
In general, measurement based models can be polynomial time reduced to the gate array model, and thus have the same power, but they are more easily parallelizable\cite{jozsa2006introduction}.


%% file citations.bib contains all the biblography
\bibliography{citations}
\end{document}
